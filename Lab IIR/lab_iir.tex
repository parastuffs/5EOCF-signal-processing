\documentclass{../template/labo}

\usepackage[utf8x]{inputenc}
\usepackage[T1]{fontenc}
\usepackage{ucs}
\usepackage{amsthm} %numéroter les questions
\usepackage{datetime}
\usepackage{xspace} % typographie IN
\usepackage{hyperref}% hyperliens
\usepackage[all]{hypcap} %lien pointe en haut des figures
\usepackage[french]{varioref} %voir x p y
\usepackage{fancyhdr}% en têtes
\usepackage[]{graphicx} %include pictures
% \usepackage{pgfplots}
\usepackage[americanresistors,siunitx]{circuitikz}
\usepackage[]{gnuplottex}
\usepackage{ifthen}
\usepackage{mathastext} % math as standfard text : units are respecting typography conventions.
\usepackage[]{subfig}
\usepackage[]{attachfile}
\usepackage{tikz}
\usetikzlibrary{babel,positioning,calc}
\usepackage{siunitx}
\usepackage{amssymb}
\usepackage{xcolor}
\usepackage{float}
\usepackage[normalem]{ulem}
\usepackage{minted}
\setminted[matlab]{
frame=lines,
framesep=2mm,
% baselinestretch=1.2,
fontsize=\small,
% linenos,
breaklines
}
\usepackage{framed}
\usepackage{charter}

%%%%%%%%%%%%
% Tables
%%%%%%%%%%%%
\usepackage{booktabs}
\renewcommand{\arraystretch}{1.1} % Opens up the table a tad
\usepackage{multicol}
\usepackage{multirow}

\newboolean{koriG}
\ifx\koriG\undefined
\correction{false}
\else
\correction{true}
\fi

\newcommand{\itgv}[1]{\ifthenelse{\boolean{corrige}}{{\color{blue}#1}}{}} %si corrigé vrai...
\newcommand{\ifgv}[1]{\ifthenelse{\boolean{corrige}}{}{#1}} %si corrigé vrai...
\newcommand{\matlab}{\textsc{matlab}}

% \correction{false}
%\correction{true}


%% fancy header & foot
\pagestyle{fancy}
\lhead{[CF5L-L1] Signal Processing\\ IIR filters}
\rhead{v1.0.0 \\ page \thepage}
\chead{\ifthenelse{\boolean{corrige}}{Corrigé}{}}
\cfoot{}
%%

\author{DLH}

\setlength{\parindent}{0pt}


\begin{document}
\tptitle{}{Labs Part II\\\vspace*{.8em}Infinite Impulse Response Filters Design}

Where FIR digital filters are \textit{finite} by definition, analogue filter know no such limitations.
Infinite Impulse Response (IIR) filters try to mimic their behaviour by having a feedback loop in the convolution product of their digital implementation.

Whilst there is are dedicated \texttt{fir1} and \texttt{fir2} \matlab functions to design FIR filters, IIR filters have no equivalent.
Instead there are individual functions for each filter type that come either in digital or simulated analogue flavours\footnote{To switch to an analogue filter, append \texttt{'s'} at the end of the function parameters.}.
Said flavours are as follows:
\begin{itemize}
  \item Bessel (analogue only): \href{https://nl.mathworks.com/help/signal/ref/besself.html}{\texttt{bessel}}
  \item Butterworth: \href{https://nl.mathworks.com/help/signal/ref/butter.html}{\texttt{butter}}
  \item Chebychev Type I: \href{https://nl.mathworks.com/help/signal/ref/cheby1.html}{\texttt{cheby1}}
  \item Chebychev Type II: \href{https://nl.mathworks.com/help/signal/ref/cheby2.html}{\texttt{cheby2}}
  \item Elliptic: \href{https://nl.mathworks.com/help/signal/ref/ellip.html}{\texttt{ellip}}
\end{itemize}

\begin{leftbar}
Explore the \href{https://nl.mathworks.com/help/signal/ug/iir-filter-design.html}{documentations} as to get a grip on how to design basic IIR filters.
\end{leftbar}

\subsection*{From FIR to IIR}

Re-take your FIR filters you have designed to clean up the audio in the previous part.
Re-implement them using IIR filters and analyse the gain in filter order for similar performance.



\subsection*{Analogue vs FIR vs IIR}

Consider a regular low-pass analogue fourth order filter with a cut-off frequency $f_c$.
Design an FIR filter that can reach the same attenuation as the Butterworth filter at $f = 10\cdot f_c$, and no more than 3dB ripple in the bandpass.
Also design an IIR filter with the same specifications.
Do so for \textbf{Butterworth} and \textbf{Chebychev} filters.

Once it's done, compare the performance of the three filters on the following criteria:
\begin{itemize}
  \item Frequency response.
  \item Attenuation at $10\cdot f_c$.
  \item Ripple in the bandpass.
  \item Ripple in the stopband.
  \item Filter order required\footnote{Only the analogue filter needs to be a 4\textsuperscript{th} order. The digital FIR and IIR filters might be different.}.
\end{itemize}



\section*{Assignment}
The assignment consist of:
\begin{enumerate}
  \item Your filter from the previous part in IIR form.
  \item The comparison between analogue, FIR and IIR filters as described in the previous section.
\end{enumerate}
You should upload your code and a maximum 3 pages report to Claco by the due date indicated on the platform.

Evaluation criteria:
\begin{itemize}
  \item The report is clear and well written. (2 points)
  \item The new IIR filter has similar performance compared to the old FIR filter and the analysis is sound. (3 points)
  \item The two times (Butterworth and Chebychev) three filters (analogue, FIr and IIR) where successfully implemented. (3 points)
  \item A interesting analysis is derived from the comparison of the filters. (2 points)
  \item The report is in English. (1 bonus point)
\end{itemize}




\section*{Resource}
\begin{itemize}
	\item \href{https://www.mathworks.com/help/signal/referencelist.html?type=function&category=filter-design&s_tid=CRUX_topnav}{Filter design functions for \matlab}
  \item \href{https://octave.sourceforge.io/signal/overview.html}{Octave signal processing functions}
\end{itemize}


\end{document}
